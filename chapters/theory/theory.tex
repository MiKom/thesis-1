\chapter{Theoretical framework for light simulation}
\label{ch:theory}

\section{Useful mathematical concepts}

\begin{df}[Spherical coordinates]
In spherical coordinate system position of a point $P$ in 3D space is described by values: $r$, $\theta$ and $\phi$, where:
\begin{itemize}
\item $r \geq 0$ (\emph{radial distance}) is the Euclidean distance from the origin to $P$.
\item $0 \leq \theta \leq \pi$ (\emph{polar angle}) is the angle between $OP$ line segment and a fixed axis ($OZ$).
\item $0 \leq \phi \leq 2\pi$ (\emph{azimuth angle}) is the angle between other axis ($OX$) and orthogonal projection of $OP$ on the reference plane ($XY$).
\end{itemize}
A point in spherical coordinates $(r, \theta, \phi)$ can be converted into Cartesian coordinates $(x, y, z)$ as follows:
\begin{eqnarray}
  x &=& r \sin \theta \cos \phi \nonumber \\
  y &=& r \sin \theta \sin \phi \\
  z &=& r \cos \theta \nonumber
\end{eqnarray}
\end{df}

\begin{df}[Solid angle]
Solid angle subtended by an object visible from point $P$ is the area of that object projected onto a unit sphere around $P$. Solid angles are measured in \emph{steradians} $[sr]$. The entire sphere subtends a solid angle of $4\pi$.
\end{df}

In this work I will, by convention, denote a unit direction vector anchored at some point $P$ in space by $\omega$. Every such vector can be expressed by a point on a unit sphere in spherical coordinates:
\begin{equation}
  \omega(\theta,\phi) := \langle \sin\theta\cos\phi, \sin\theta\sin\phi, \cos\theta \rangle.
\end{equation}
Consider a differential surface element $dS$ on a unit sphere spanning from $\theta$ to $\theta + d\theta$ and from $\phi$ to $\phi + d\phi$. Differential area of a set of directions $d\omega$ (or, equivalently, solid angle subtended by $dS$) is then
\begin{equation}
\label{eq:dw2spherical}
  d\omega = \frac{\sin\theta d\theta d\phi}{r^{2}} = \sin\theta d\theta d\phi.
\end{equation}
This relationship is the key to converting integrals over solid angles to integrals over directions.

\begin{df}[Projected solid angle]
Projected solid angle \parencite{nicodemus76} of an object $S$ measured from point $P$ is the solid angle of $S$ projected onto a unit disk perpendicular to the surface normal vector $\vec{N}(P)$ at $P$. Since $\vec{N}(P)$ defines the orientation of the projection plane the projected solid angle is cosine-weighted solid angle of $S$:
\begin{equation}
  \omega^{\perp} = |\cos\theta| \cdot \omega ,
\end{equation}
where $\theta$ is the angle between $\omega$ and surface normal at $P$. 
\end{df}

\begin{df}[Dirac's delta]
  Dirac's delta function \parencite{dirac58} is a generalized function defined over real numbers that is zero everywhere except at the origin. It can be thought of as an infinitely high, infinitely thin spike at $x=0$ with unit area under the curve.
\begin{equation}
  \int_{-\infty}^{\infty} \delta(x) dx = 1.
\end{equation}
One particularly useful property of the delta function is
\begin{equation}
  \int_{-\infty}^{\infty} \delta(x - x_{0}) f(x) dx = f(x_{0}).
\end{equation}
\end{df}

\section{Simulation assumptions}
\label{sec:assumptions}
Simulating light propagation in 3D scenes is very computationally intensive task. It is generally reasonable to make certain assumptions about the nature of light or the scene itself that can drastically reduce algorithm complexity and speed up simulation without sacrificing noticeable image quality. Some physical phenomena are insignificant at macroscopic scale or their effects are too minuscule relative to added simulation complexity.

The following properties are implicitly assumed for the rest of this work:
\begin{itemize}
\item Light transport is considered in the context of \emph{geometric optics} on macroscopic scale (quantum effects are ignored).
\item Light travels in vacuum with no participating media and is unpolarized.
\item Light consists of radiation of mixed wavelengths and spectral effects, such as dispersion, are ignored.
\item The scene is in the state of energy equilibrium. 
\end{itemize}

\section{Radiometric quantities}
The natural basis for study of light transport in computer graphics is the field of \emph{radiometry} and, by lesser extent, \emph{photometry}.

Radiometry is the study of electromagnetic radiation (which, of course, includes visible light spectrum) and its interaction with matter. Photometry, on the other hand, studies human perception of light. Defining key radiometric quantities is necessary for proper understanding of light transport theory.

Unless noted otherwise definitions in this section are based on chapter one of \cite{mccluney94}.

\begin{df}
Radiant power $[W=J \cdot s^{-1}]$ (or \emph{radiant flux}) is total amount of energy passing through a surface per unit time.
\begin{equation}
  \Phi := \frac{dQ}{dt}
\end{equation}
\end{df}

\begin{df}
Intensity $[W \cdot sr^{-1}]$ is radiant flux density per unit solid angle.
\begin{equation}
  I := \frac{d\Phi}{d\omega}
\end{equation}
It describes directional distribution of light and is only meaningful for point or small light sources.
\end{df}

\begin{df}
Irradiance $[W \cdot m^{-2}]$ is  area density of radiant flux arriving at a surface around point $x$.
\begin{equation}
  E(x) := \frac{d\Phi(x)}{dA(x)}
\end{equation}
Although formal definition of irradiance includes both sides of the surface in the measurement, for the purpose of computer graphics it is more convenient to consider only the side with normal vector $N(x)$ pointing outwards, thus irradiance measures incident radiant flux in the upper hemisphere around point $x$.
\end{df}

\begin{df}
Radiant exitance $[W \cdot m^{-2}]$ \parencite{nicodemus78} is area density of radiant flux leaving a surface around point $x$.
\begin{equation}
  M(x) := \frac{d\Phi(x)}{dA(x)}
\end{equation}
Radiant exitance is analogous to irradiance but measures outgoing radiant flux around point $x$.
\end{df}

\begin{df}
Radiance $[W \cdot m^{-2} \cdot sr^{-1}]$ is radiant flux density per unit area per unit solid angle. It is measured on a differential surface perpendicular to $\omega$ and around point $x$.
\begin{equation}
  L(x,\omega) := \frac{d^{2}\Phi(x, \omega)}{dA(x) |\cos\theta| d\omega},
\end{equation}
where $\theta$ is the angle between $\omega$ and surface normal $\vec{N}(x)$. Radiance can also be formulated by using the dot product of $\omega$ and surface normal:
\begin{equation}
  L(x, \omega) := \frac{d^{2}\Phi(x, \omega)}{|\omega \cdot \vec{N}(x)| dA(x) d\omega}.
\end{equation}
\end{df}

It is often useful to distinguish between radiance arriving at a surface and radiance leaving a surface. Throughout this work \emph{incident radiance} will be denoted by $L_{i}(x, \omega)$ while \emph{exitant radiance} will be denoted by $L_{o}(x, \omega)$. Following from assumptions in section \ref{sec:assumptions} it is easily seen that:
\begin{equation}
  L_{i}(x, \omega) = L_{o}(x, -\omega).
\end{equation}

\section{The BRDF}
The next step in formulating mathematical model for light transport is the introduction of the \emph{bidirectional reflectance distribution function} or BRDF for short. The formulation here is based on \cite{veach97}.

Let $x$ be a point on a surface with corresponding normal vector $N(x)$. Consider an outgoing direction $\omega_{o}$ and exitant radiance $L_{o}(x, \omega_{o})$. For a fixed incoming direction $\omega_{i}$ the irradiance due to incident light from an infinitesimal cone $d\omega_{i}$ around $\omega_{i}$ is
\begin{equation}
  dE(x, \omega_{i}) = L_{i}(x, \omega_{i}) |\cos\theta_{i}| d\omega_{i}.
\end{equation}
Recall from section \ref{sec:assumptions} that the modelling of light in this work assumes principles of geometric optics, that includes linearity of light. From this assumption it follows that exitant radiance at $x$ in direction $\omega_{o}$ is linearly proportional to the irradiance
\begin{equation}
  dL_{o}(x, \omega_{o}) \propto dE(x, \omega_{i}).
\end{equation}

\begin{df}[BRDF]
Bidirectional reflectance distribution function $[sr^{-1}]$ is the constant of proportionality of outgoing radiance due to irradiance at point $x$ on some surface.
\begin{equation}
  f_{r}(\omega_{o}, \omega_{i}) := \frac{dL_{o}(x, \omega_{o})}{dE(x, \omega_{i})} = \frac{dL_{o}(x, \omega_{o})}{L_{i}(x, \omega_{i}) |\cos\theta_{i}| d\omega_{i}}.
\end{equation}
\end{df}

In other words BRDF is radiance leaving in direction $\omega_{o}$ per unit of irradiance arriving from $\omega_{i}$. Alternative notation for the BRDF function is $f_{r}(\omega_{i} \rightarrow \omega_{o})$.

Physically based BRDFs have the following properties:
\begin{itemize}
\item \emph{Reciprocity}: For all pairs of directions $\omega_{i}$, $\omega_{o}$: $f_{r}(\omega_{o}, \omega_{i}) = f_{r}(\omega_{i}, \omega_{o})$.
\item \emph{Energy conservation}: The total energy of reflected light is less than or equal to the energy of incident light.
\end{itemize}

\begin{df}
Directional--hemispherical reflectance \parencite{sillion94} is the fraction of incident radiant flux density in a given direction that is reflected over the entire hemisphere of possible directions.
\begin{equation}
\label{eq:dhrefl}
  \rho_{dh}(\omega_{i}) = \int_{\Omega} f_{r}(\omega_{o}, \omega_{i}) |\cos\theta_{o}| d\omega_{o}
\end{equation}
\end{df}

\section{Fresnel reflectance}
Fresnel equations describe the amount of incoming light that is reflected off a surface. Those equations are different depending on whether given surface is a dielectric or a conductor and also depend on direction of light polarization.

\begin{df}[Snell's law]
Snell's law describes the relationship between the angle of incident light and the angle of transmitted light on the surface boundary between media. It is of the form:
\begin{equation}
  \eta_{i} \sin\theta_{i} = \eta_{t} \sin\theta_{t},
\end{equation}
where $\eta_{i}$, $\eta_{t}$ are indices of refraction of incident and transmitted media and $\theta_{i}$, $\theta_{t}$ are the corresponding angles between surface normal vector and incident and transmitted direction vectors.
\end{df}

Recall from section \ref{sec:assumptions} that light is assumed to be unpolarized. To get \emph{Fresnel reflectance} for unpolarized light it is sufficient to take the average of Fresnel reflectance for each polarization direction.

\begin{df}
Let $r_{\parallel}$ be the Fresnel reflectance for parallel polarized light and $r_{\perp}$ be the Fresnel reflectance for perpendicular polarized light. Reflectance for unpolarized light is then:
\begin{equation}
  F_{r} = \frac{1}{2} (r_{\parallel} + r_{\perp}).
\end{equation}
\end{df}

Dielectrics reflect a fraction of incident light and transmit the rest. For physically based Fresnel reflectance $F_{r} \in [0,1]$ thus the amount of reflected light is $F_{r}$ and the amount of transmitted is $1 - F_{r}$ as a consequence of conservation of energy.

\begin{df}[Fresnel reflectance for dielectrics]
An approximation of Fresnel reflectance for dielectric surfaces \parencite{phar2010} is of the form:
\begin{eqnarray}
r_{\parallel} &=& \left( \frac{\eta_{t} \cos\theta_{i} - \eta_{i} \cos\theta_{t}}{\eta_{t} \cos\theta_{i} + \eta_{i} \cos\theta_{t}} \right)^{2} \\
r_{\perp} &=& \left( \frac{\eta_{i} \cos\theta_{i} - \eta_{t} \cos\theta_{t}}{\eta_{i} \cos\theta_{i} + \eta_{t} \cos\theta_{t}} \right)^{2},
\end{eqnarray}
where $\eta_{i}$ and $\eta_{t}$ are indices of refraction of incident and transmitted media, $\theta_{i}$ is the angle between surface normal and incident direction $\omega_{i}$ and $\theta_{t}$ is the angle between surface normal and transmitted direction $\omega_{t}$ computed with Snell's law.
\end{df}

Conductors, on the other hand, reflect a fraction of incident light absorbing the rest. The absorbed light is turned into heat.

\begin{df}[Fresnel reflectance for conductors]
An approximation of Fresnel reflectance for conducting surfaces \parencite{phar2010} is of the form:
\begin{eqnarray}
  r_{\parallel} &=& \frac{(\eta^{2} + k^{2}) \cos^{2}\theta_{i} - 2\eta\cos\theta_{i} + 1}{(\eta^{2} + k^{2}) \cos^{2}\theta_{i} + 2\eta\cos\theta_{i} + 1} \\
  r_{\perp} &=& \frac{(\eta^{2} + k^{2}) - 2\eta\cos\theta_{i} + \cos^{2}\theta_{i}}{(\eta^{2} + k^{2}) + 2\eta\cos\theta_{i} + \cos^{2}\theta_{i}},
\end{eqnarray}
where $\eta$ is the index of refraction of the conductor, $k$ is the \emph{absorption coefficient} and $\theta_{i}$ is the angle between incident direction $\omega_{i}$ and surface normal.
\end{df}

\section{Reflection models}

\subsection{Lambertian reflection}
The simplest, yet widely used, reflection model is \emph{Lambertian diffuse reflection}. Lambertian surfaces scatter incident light uniformly in all directions. While such ideal diffuse reflection is physically implausible, in practice it is fairly good approximation of many real life materials like matte plastic or clay.

Let $f_{d}$ be the BRDF of perfectly diffuse surface. Due to BRDF reciprocity $f_{d}(\omega_{o}, \omega_{i}) = f_{d}(\omega_{i}, \omega_{o})$, therefore $f_{d}$ is independent of both $\omega_{i}$ and $\omega_{o}$ and thus constant:
\begin{equation}
  f_{d}(\omega_{o}, \omega_{i}) = c.
\end{equation}
Substitution into equation (\ref{eq:dhrefl}) yields \emph{diffuse reflectance} coefficient:
\begin{eqnarray}
  \rho_{d} &=& c \int_{\Omega} |\cos\theta_{o}| d\omega_{o} \\
  &=& c \int_{0}^{\pi} \int_{0}^{2\pi} |\cos\theta_{o}| \sin\theta_{o} d\theta_{o} d\phi_{o} \\
  &=& c \pi.
\end{eqnarray}
In terms of diffuse reflectance the Lambertian BRDF is then
\begin{equation}
  f_{d}(\omega_{o}, \omega_{i}) = \frac{1}{\pi} \rho_{d}.
\end{equation}

\subsection{Specular reflection}
Another useful reflection model is \emph{perfect specular reflection}. Perfect specular surfaces are ideally smooth mirrors reflecting light in a single direction. Let $\omega_{i}(\theta_{i}, \phi_{i})$ be the incident light direction and let $\omega_{o}(\theta_{o}, \phi_{o})$ be the outgoing (reflected) light direction. Perfect specular reflection implies that:
\begin{eqnarray}
\label{eq:reflection}
  \theta_{o} &=& \theta_{i} \nonumber \\
  \phi_{o} &=& \phi_{i} \pm \pi.
\end{eqnarray}

In terms of radiance, for physically based BRDF describing specular reflection \parencite{phar2010}:
\begin{equation}
  L_{o}(\omega_{o}) = f_{r}(\omega_{o}, \omega_{i}) L_{i}(\omega_{i}) = F_{r}(\omega_{i}) L_{i}(\omega_{i}),
\end{equation}
where $F_{r}$ is Fresnel reflectance. Since $\theta_{o} = \theta_{i}$, therefore $F_{r}(\omega_{i}) = F_{r}(\omega_{o})$.

Let $f_{s}$ be the BRDF of perfectly specular surface. $f_{s}$ should have non-zero value only for $\omega_{i}$, $\omega_{o}$ direction pairs satisfying the condition imposed by equation (\ref{eq:reflection}). This can be expressed implicitly using a product of delta functions \parencite{cohen93}:
\begin{eqnarray}
  f_{s}(\omega_{o}, \omega_{i}) &=& f_{s}(\theta_{o}, \phi_{o}, \theta_{i}, \phi_{i}) \nonumber \\
  &=& F_{r}(\theta_{i}, \phi_{i}) \frac{\delta(\cos\theta_{i} - \cos\theta_{o})}{|\cos\theta_{i}|} \delta(\phi_{i} - (\phi_{o} \pm \pi)).
\end{eqnarray}

Since $f_{s}$ is not a well-defined function it requires special handling in the implementation.

\section{The light transport equation (LTE)}
The light transport equation, sometimes also called the rendering equation, has been originally formulated by \emph{James T. Kajiya} in \cite{kajiya86}. It describes the equilibrium distribution of radiance in a scene. For every point it gives total reflected radiance in terms of emission from the surface, its BRDF and the distribution of incident illumination. Good formulation of the LTE is given, for example, in \cite{phar2010}:
\begin{equation}
  L_{o}(x, \omega_{o}) = L_{e}(x, \omega_{o}) + \int_{\Omega} f_{r}(x, \omega_{o}, \omega_{i}) L_{i}(x, \omega_{i}) |\cos\theta_{i}| d\omega_{i},
\end{equation}
where
\begin{itemize}
\item $L_{o}(x, \omega_{o})$ is the total outgoing radiance in direction $\omega_{o}$ from point $x$.
\item $L_{e}(X, \omega_{e})$ is the outgoing radiance from point $x$ due to surface emission.
\item $L_{i}(X, \omega_{i})$ is the total incident radiance on point $x$ from the hemisphere of direction $\Omega$ around that point.
\item $f_{r}(x, \omega_{o}, \omega_{i})$ is the value of the BRDF of surface at point $x$ given incident and outgoing direction vectors.
\item $\theta_{i}$ is the angle between incident radiance direction $\omega_{i}$ and surface normal at point $x$.
\end{itemize}

Solving the LTE is an area of active research in computer graphics. It cannot be computed analytically for any non-trivial scene and high dimensionality of the integral causes most of the numerical integration methods to converge very slowly to the approximate solution.

One viable technique of solving the LTE is \emph{Monte Carlo integration}, which will be explored in the next chapter.
